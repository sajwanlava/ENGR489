\chapter{Evaluation}\label{C:Evaluation}

\par The contents of this chapter will, compare and contrast the chosen methodology to others, justify the processes used within the Grounded Theory framework, evaluate the overall theory using Glaser's criteria and evaluate the findings against existing literature.

\section{Chosen Methodology}

Alternative qualitative methodologies were not explored at the beginning of this project. Two alternatives are compared with the project methodology to evaluate the chosen methodology. 

\subsection{Ethnography}

Ethnography is the study of people or "folk" within a group \cite{ethno}. It is an intensive study-type as participants need to be regularly contacted, and different forms of data collection methods should be used from interviews to surveys to observations \cite{ethno}. In regards to findings, this methodology focuses explicitly on culture \cite{ethno}. Usually, these occur in-person; this is, so the researcher can understand the behaviours and view-points of the participants clearer \cite{ethno}.  While predominantly qualitative, quantitative methods can be employed to support the other type of data \cite{ethno}. 
\newline
\par 
Ethnography would have fit to the initial brief of the project. Though, while Grounded Theory aims to present a new theory, Ethnography aims to find patterns between the ethno (folk) within the sample \cite{ethno}. This would have provided no potentially new insights, and it would have focused on the culture of programmers rather than other traits. Ethnography would not have been suitable to the COVID-19 situation as the data gathering process calls for interviewers to interact with the participants directly \cite{ethno}. This is something that would have been impossible to do in the earlier half of the year. Another additional negative to this approach is that long-term interaction and relationships with participants may introduce biases in the findings; something which Grounded Theory aims to minimise. Biases can skew results which make them more inaccurate, and consequently, results can be considered weak. 
 \newline
 \par
 A benefit of ethnography compared to Grounded Theory is that there is a lot more data being provided to the study from a lot of different inputs such as interviews and video recordings \cite{ethno}. This can strengthen findings. However, this also means that the pattern formation is far more robust than the already lengthy process of Grounded Theory analysis, which makes ethnography best suited for a longer-term project; not just two trimesters. 

\subsection{Phenomenology}

Phenomenology is the study of human perception and relies on the understanding of participant experiences \cite{ethno}. In this methodology, interviews are the only source of data \cite{ethno}. The interviews are transcribed and studied by the researcher \cite{ethno}. This is so that the researcher can understand the full narratives of participants. Researchers then can pull out key points from the transcripts. As these critical points get added upon as more interviews take place, a structure is defined. This structure is then reduced to the fundamental aspects and used to describe the phenomena \cite{ethno}. Researchers are then able to go back to interview participants for more data gathering and confirmed approval of the outputted description \cite{ethno}. 
\newline
\par
Phenomenology would have fit to the initial brief of the project as it a study of experiences, in this case, the experiences of programmers in regards to security practices. However, while it only aims to define practices towards the development of a theory, Grounded Theory aimed to provide a new theory and consequently newer insights and ideas. Phenomenology would have been suitable for the lockdown situation we had at the beginning of the year as interviews could have occurred conducted over Zoom, much like the ones done for this project and under the Grounded Theory methodology. 

\subsection{Comparison to the Chosen Methodology}

\par This project followed a Grounded Theory methodology using interviews and observations for gathering data, as was outlined in Ch.3. Critical factors for choice were that it has strength for exploring human and social aspects of concepts \cite{geeks}. Specific to influences to security practices, it was essential to have a qualitative method as this allows for the intake of observations and answers as supporting evidence \cite{geeks}. It also aims to focus more on this collection of data, rather than existing knowledge on the topic to minimise any biases. Above all, Grounded Theory allows for new theory formation to gain new insights and perspectives in the area of secure programming \cite{geeks}.
\newline
\par
Compared to the two alternatives which were explored in this section, Grounded Theory methodology was best suited to the initial brief. It aimed to provide a theory while ethnography and phenomenology seem to be methodologies which can support theory building. Ethnography focuses on one aspect, culture, as the overarching topic of the research questions, and phenomenology aims to define existing experiences concisely. However, ethnography could have been an exciting methodology to pursue given a more extended period as it pulls in various sources of data and allows for the researcher to be immersed in the lives of participants to gain a better understanding of their answers.  

\section{Internal Methodology Processes}

This section evaluates the participant recruitment and data gathering strategy used in this project. It outlines what did well and why, and what could have been improved. 

\subsection{Participant Recruitment}

While the recruitment aimed to find people from Linkedin and Meetup groups such as OWASP NZ, this became difficult. The constant recruitment across Meetup groups without actually going to Meetups (despite the COVID-19 lockdown) made it, so my account was marked as spam, and I was unable to post anymore. Linkedin proved to be ineffective as well as groups do not often show notifications of new posts to users by default. The topic of security also made people hesitant to participate, and so did the disruptions in people's lives due to the pandemic. Therefore, much recruitment was done by asking family friends and friends. This also made it comfortable to ease into the data gathering process.
\newline
\par\textbf
Using family friends and friends for this investigation did not skew or introduce biases to the results, which was the initial worry. This was because everyone came from a range of study backgrounds, had varying years of experiences, and worked across the industry in different roles. With the addition of other individuals as a part of this sample, and expanding it to all of New Zealand, the theory had a depth to it as it could make connections from a wider group of people.


\subsection{Data Collection}
Interviews were chosen as the method of data collection as this is what is the norm in Grounded Theory studies \cite{geeks}. Interviews are semi-structured and are more open-ended compared to the likes of a diary study or a survey. This is beneficial as it allows for participants to feel more comfortable and have more control over their answers. It also allows an interviewer to correspondingly make observations of the participant and make connections with their responses \cite{geeks}. A more concrete emergent theory can form as it is built upon two sources. However, this also can introduce biases by the researcher as observations can hold stereotypes. 
\newline
\par
 As this year was unprecedented, very early on the data gathering process shifted to be one predominantly online. Only two participants were interviewed in person; \textit{[P6]} and \textit{[P14]}. This meant that robust observations could not be taken of their ways of working in the office (most were working from home), but some could still be undertaken through Zoom. Facial expressions and tone of voice were big indicators as the face was the focus of the video. From this, the observation about the differences in attitudes of developers and that of more security-related roles such as architects and analysts was able to be made. 
\newline
\par 
Fifteen semi-structured interviews occurred. For two trimesters worth of work, it was enough, and it had also hit the saturation point where a minimal amount of new data was added to the theoretical findings. Appendix D provides the interview template that was used. It shows the four primary areas of questioning. The participant background was the only section which remained consistent across all interviews, and the rest changed. The other three sections provided questions that could have been asked to a person dependent on the participant background, but not all were at each time. Questions were also adapted dependent on the participant's ability to answer the questions. Due to the nature of the topic, there were some privacy concerns, and often the field of the organisation meant that participants had signed non-disclosure agreements as part of the contract which meant that some questions remained unanswered.  
\newline
\par
After completing the data gathering process,  it would have been good to perhaps split current work into two sections, one on the more "hard" influences and another on the "soft" influences. This is so more answers could be prompted pertaining to technical impacts on participants. While Grounded Theory aims to leave questions open-ended, the specific section would have been a benefit regardless as it would have provided the theory with more fresh and new inputs. 

\section{Glaser's Criteria}

There are standards in which an emergent Grounded Theory is assessed for quality; fit, work, relevance, modifiability \cite{crit}. These criteria will be used to assess this project's theory. 

\subsection{Fit}

\textit{"Fit refers to the emergence of conceptual codes and categories from the data rather than the use of preconceived codes or categories from extant theory." \cite{crit}}
\newline
\par 
This criterion means that findings should not be based on existing theories. The theory presented represented this because prior robust background reading had not occurred and the related works described in the background chapter were to provide an understanding on types of qualitative studies that could be undertaken rather than drawing from the conclusions. The result of the theory was different from what was expected, less technical reasons for the influences on security, which meant that there was a degree of unknowing starting this research. Numerous quotes were also provided with to further support that new data had been used. 

\subsection{Work}

\textit{"Work refers to the ability of the grounded theory to explain and interpret behaviour in a substantive area and to predict future behaviour." \cite{crit}}
\newline
\par 
This criterion means that the theory's categories and connections should be explained. The Theory chapter describes these in detail and does so by providing further support by the use of numerous quotes from participants. Future behaviour is mentioned briefly in Limitations (sec 6.1) when discussing the shift to DevOps, more technical-oriented training and evolving technologies. It is also further supported when comparing the theoretical findings to literature in the section after this (sec 5.4).

\subsection{Relevance}

\textit{"Relevance refers to the theory’s focus on a core concern or process that emerges in a substantive area. Its conceptual grounding in the data indicates the significance and relevance of this core concern or process thereby ensuring its relevance." \cite{crit}}
\newline
\par 
This criterion means that the theory must reference the core question and the object area of study. This is done by the categories of the theory answering the initial question; Why do programmers do what they do? The categories are also interlinked, which shows the process to the reasons as to why security decisions are made. 

\subsection{Modifiability}

\textit{"Modifiability refers to the theory’s ability to be continually modified as new data emerge to produce new categories, properties or dimensions of the theory. This living quality of grounded theory ensures its continuing relevance and value to the social world from which it has emerged."\cite{crit}}
\newline
\par 
This criterion means that the theory should be open to adaption as more data gets gathered. This was proven in the data collection and analysis as when more people were interviewed, the gathered data was able to be analysed and became codes. The modifiable aspect of this was the continuous editing to the categories if they were not considered viable to the theory as more interviews were occurring—also, the changes to the questions during and in between interviews. 

\section{Literature Comparison}

This section compares the theory to existing published literature. 

\subsection{Corporate hackathons, how and why? A multiple case study of motivation, projects proposal and selection, goal setting, coordination, and outcomes}

The investigation outlined in this report followed five teams in a sizeable corporate hackathon. Researchers  Pe-Than,  Nolte, Filippova, Bird, Scallen and Herbsleb interviewed all team members immediately before, immediately after and four months afterwards the hackathon, and during the event, they simply made observations \cite{corp}. They focused on three research questions \cite{corp}:
\newline
\par
\textbf{R1} - \textit{"what were the team processes, and how did they differ between PETs and FTs?  "}
\newline
\par
\textbf{R2} - \textit{"what were the conditions that contributed to sustaining the projects after the event?"}
\newline
\par
\textbf{R3} - \textit{"what impacts did participants believe the event had on them?"}
\newline
\par
There question most related to the emergent theory is research question three, which is the one that will be compared to the general views of internal hackathons from the category aspect explanations in chapter four. The "perceived impacts to individuals skills" section in "Corporate hackathons, how and why? A multiple case study of motivation, projects proposal and selection, goal setting, coordination, and outcomes" outlines that participants became more confident in their skills as they learnt more, the majority also enjoyed themselves \cite{corp}. The participant's attitudes had also changed to become more positive and confident about learning new skills by themselves \cite{corp}. This relates to the theory as the organisation's category impacts culture and they correspond to the aspects of security education practices and biases and attitudes. Aspects in the emergent theory also described how people managed to learn new skills as they were applying them practically and learning as they went; this was described in the research done by Pe-Than,  Nolte, Filippova, Bird, Scallen and Herbsleb \cite{corp}.
\newline
\par
The idea of internal, corporate hackathons is relatively new as research is limited. However, it is prevalent in the industry in the United States of America. As New Zealand matches international trends, it will start to become more popular domestically as well. 

\subsection{Challenging Software Developers: Dialectic as a Foundation for Security Assurance Techniques}

In "Challenging Software Developers: Dialectic as a Foundation for Security Assurance Techniques"\cite{charles}, the authors Weir, Rashid and Noble identify that security is more reliant on developers, but that the developers are not providing the security that is needed. Similar to the study outlined in the ENGR489 study, this was conducted using a Grounded Theory Methodology. In contrast, there were two parts of the survey, and each one was a separate grounded theory mini-study. 
\newline
\par The authors' interview participants in order to obtain data so they can find ways to "help programmers themselves to improve security given existing constraints".  
\newline
\newline
The two major findings of each survey showed:

\begin{enumerate}
\item Developer security is based on challenges in order to motivate better practice. These challenges are often fun adversary questioning usually to do with review and advisory.  This emerged as the core theory as it was interwoven through most of the participant responses.
\item Six assurance techniques were identified in being the most helpful; threat assessment, stakeholder negotiation, configuration review, vulnerability scan, source code review and penetration testing. They all help provide software security. 
\end{enumerate}

\par These two ideas are linked, as not only do those six assurance techniques mitigate any challenges, they also provide challenges to the developer when dealing with them. 
\newline
\par This paper was valuable to read as it provided another method of conducting grounded theory; by the use of two separate surveys that can be combined to provide their own intermingled findings. There are also similarities to the topic of this ENGR489 project. In a direct comparison, the authors outlined that they wanted to help programmers improve their security, the project outlined in this final report was initially more focused on the improvement of the technical design of security tools, frameworks and libraries. Ultimately, as the categories found in the emergent theory are defined as culture, trends and organisations, these are all aspects which influence a programmer in terms of security, so combined with the findings of "Challenging Software Developers: Dialectic as a Foundation for Security Assurance Techniques", perhaps a unique and robust security education in the workplace can be outlined. 
\newline
\par

\subsection{A Survey on Developer-Centred Security}

\par Tahaei and Vaniea undertake an extensive literature review of 49 publications on security studies with participants who were software developers \cite{dev}. They then present an overview of the methodologies and current research in the area \cite{dev}. This subsection will focus on the latter. 
\newline
\par 
There were eight significant themes in the results shown by the authors. They were "Organisations and Context", "Structuring Software Development", "Privacy and Data", "Third Party Updates", "Security Tool Adoption", "Application Programming Interfaces (API's), "Programming Languages" and "Testing Assumptions" \cite{dev}. Organisations and Context focuses on developers working within the context of organisations, teams and cultures \cite{dev}. They are inline with the ways of working of such constraints which in turn impacts the development \cite{dev}. This lines with the theory as one of the categories is organisations, and another is culture. "A Survey on Developer-Centred Security" specifically mentioned dedicated security teams and security-oriented organisations \cite{dev}. As described by the authors the dedicated security teams are a deterrence as developers still are not impacted by them as much as teams are small and not involved with an entire project \cite{dev}. This was mentioned in a category in the presented theory. Security oriented organisations pertain to security being involved in all stages of development, and it is cited as being a benefit. This was another aspect of the theory which was mentioned. 
\newline
\par 
The section "Programming Languages" also stated that often developers do not have free reign on programming languages able to be used \cite{dev}. This was exhibited in the findings of this project's theory in "Technology Stack". Participants of the ENGR489 project stated that often they use what is already established and are not given flexibility in choice, which ultimately can affect the security libraries, tools and frameworks in place. 
