\chapter{Work Completed}\label{C:Completed}

\par As of yet, the progress made on this study has been predominately focused on the human ethics application as well as the steps in crafting appropriate interview questions. Before any contact with potential interview participants, the human ethics application has to be approved. This is to ensure that research follows responsible and robust processes involving people and their data, consequently, proceeding ethically. As this has been in a recursive stage of editing and sending back to be reviewed, there has been iterative work designing potential questions for the semi-structured interview. Processes which have involved pilot studies in order to focus on an area of my topic. As the human ethics application is yet to be approved, my data collection and analysis stages have not started. Each of the areas are discussed in detail in the following sections. 

\section{Human Ethics Application}

\par There was an immediate pressure to complete the human ethics application as the rest of the project is highly dependent on the participant interviews. This meant that the research had to be planned quite quickly in order to provide the human ethics committee with all the relevant information needed to approve the application. The initial application was submitted 8th of April. It has since undergone pre-committee revisions and post-committee revisions and is currently pending a second committee review. This application outlined a series of multichoice and short-answer questions which had to be answered thoroughly and supporting documents had to be submitted. These supporting documents included the participant information sheet, consent to interview document, an interview guide, and as this research’s recruitment will be done through mailing lists and Meetup and LinkedIn posts, a sample recruitment post was also supplied. 
\newline
\par The human ethics application specified key methods for ethical data collection in order to maintain a research that followed ideal  human ethics policy
(https://www.wgtn.ac.nz/documents/policy/research-policy/human-ethics-policy.pdf
). The application went into details of data collection and recruitment, conformation with the university Te Tiriti o Waitangi statute, project risks, and data management. Answering these sections meant that a lot of planning has gone into the interview process for data collection. In particular it was decided that the selection process will follow purposeful sampling for the interviews. This is because participants will need to be selected on whether they are professional programmers in industry. The study will ideally include participants with a range of job titles and years in the field in order to find interesting comparisons during the interview process in the way security practices are adopted in the developed software. Participants will consequently be filtered by appropriate job titles, and be older by the age of 18 to avoid complexities with the Vulnerable Children Act 2014
(http://www.legislation.govt.nz/act/public/2014/0040/latest/DLM5501618.html?src=qs
). 
\newline
\par Focus on the Te Tiriti o Waitangi was necessary as this is a research project run in New Zealand with participants who work in the domestic industry. Therefore, the recruitment of participants and the conducting of the interviews will be run in a way which respects the principles outline in the university statute, which derive directly from Te Tiriti o Waitangi. Specific care was taken to this as there is only a certain threshold that this study can pertain to. However, it was decided that through my recruitment, I should work under the principle of Whai wāhi (participation) and encourage the participation of tangata whenua by advertising it on accessible platforms and to organisations with more of an emphasis of Te Tiriti in their core values. An alum of Maori decent was also consulted on this and they mentioned that in the participant information and consent forms, it should be explicitly clear that participants still have ownership of their data and are able to edit and remove information as they wish. 
\newline
\par Corresponding to the point above, the interviews will be confidential rather the anonymous to allow for participant emendations and to also allow for follow up questions if necessary. They will in no way be able to be identified and in any report outputs of this study will not name participants and instead be either referred to by role or by pseudonyms. Data will also be aggregated to maintain confidentiality. Participants will also be reminded before the interview as well as in the information and consent documents, that they should not divulge any sensitive information. 
\newline
\par Due to complications with COVID-19 the interview will also be adapted to allow for online capabilities using applications such as Zoom. There were issues surrounding obtaining consent, and the application had to be amended to allow for electronic consent. This will be done by emailing the form to participants, and consent will be obtained by a simple email back of agreement. Electronic signing of the form was not applicable as not everyone may have access to software with the capabilities to do so. It was also decided that the interview will offer a gift reward for participation for their time during the pandemic and this will be a supermarket voucher. Safety was consequently paramount in the planning of the pandemic for both the interviewer and the interviewee. In the case of any in-person interviews sanitiser and a box of tissues will be in hand as well as disinfection of the meeting rooms before and after interviews, and the interviews themselves will be conducted while maintaining social distancing practices. To thank participants for their time, they will be offered a gift voucher to a supermarket.



\section{Crafting the Semi-Structured Interview}

\par In the ethics application planning had to occur well ahead of interviews and there was a focus on creating the questions. However, as grounded theory involves a semi-structured interview process, a guide was submitted alongside the application. This is because questions are meant to change dependent on the participant as well as the narrowing focus on the ultimate theory. The interviews will also include open-ended questions which allow for more personal, in-depth responses which can hold more information to be analysed. It is expected that the analysis will occur immediately after interviews in order to revise questions before the next participant.  All interviews are expected to take a minimum of 30 minutes, and will be transcribed. This transcription will be available for participants at their request for any amending of content
\newline
\par Interviews will be expected to start by collecting information on participant background. This means groupings can happen by education, roles and experience. The next section of the interview will cover their current security practices that they implement in their code. This will then lead on to the impacts of security and languages in the projects they have worked on, and any background in the testing of those practices and successes of some over others. This sections allow for the simplest of aggregations to occur when all data has been collected and analysed. 
\newline
\par Potential interview questions have been planned alongside these sections. However, they are too broad and are disjointed of each other; often focusing on several different aspects of security practices. More thought is going into these questions as having a focus now, will provide the semi-structured interview the flow it needs to collect concise data. This is occurring through a small pilot study.




