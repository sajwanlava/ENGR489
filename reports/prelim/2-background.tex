\chapter{Background}\label{C:Background}

\par Conceptually, this project aims to gather data in order to ultimately form a theory. Therefore, It is essential that I know enough information about the topic of security so that I am able to interview professional engineers in a manner which is knowledgeable. This chapter will include background on the importance of information security practices, what practices are widely used, summaries on existing research on similar subject matter, and a discussion on the chosen methodology.  

\section{Research Methodology}

Grounded theory (GT) is the chosen methodology for this study. It was an appropriate choice as the study focuses on human aspects and GT is appropriate to study these as it is way of analysing qualitative data with the end-goal of defining a new theory from the sampled data \textbf{(GROUNDED THEORY FOR GEEKS)}. There are several procedures in the GT  methodology in order to make the final theory.  The theory is expected to be explanatory, focusing on describing elements of the findings rather than stating. It should also be based on the collected responses and observations, rather than pre-conceived ideas. Avoidance of an extensive literature review should occur accordingly.
\newline
\par Initially, researchers must choose a topic and research whether the Grounded Theory method is the right one for the research project \textbf{(GROUNDED THEORY FOR GEEKS)}. After contacting sample groups the iterative process of data collection occurs. Questions adapt between collection as analysis of the previous answers occurs. This refinement of questions happens to delve deeper into the traits of the emerging theory. When saturation is achieved, no more new findings are displayed which contribute to the final theory. This results in the “grounded” nature of the overall theory \textbf{(Strauss and Corbin (1990))}. 

\section{Information Security in New Zealand}

Information security is defined by three objectives; confidentiality, integrity and availability, most commonly referred to as the CIA triad. Confidentiality is the act of keeping information private to protect information, Integrity is the act of maintaining information validity and Availability is the act of ensuring reliable access to information.  Any breach of those principles can result in significant negative impacts economically, and to personal emotional and physical well-being. 
\newline
\par A study run by Aura Security showed \textbf{(Aura Information Security)} showed a 10\% increase in cyber-attacks on New Zealand businesses between the years of 2018-2019. In NZ the rise is attributed to the digitisation of day-to-day way-of-life; as new technologies are continuously developed there is an urgency to deploy products. There is a lot of pressure on programmers to finalise these products and the deviation of attention to the finished output means that there are many new threats to security. \textbf{(New Zealand’s cyber security strategy 2019 )}. Malicious attackers are also becoming more sophisticated. Individual threat attackers now seem to have the same knowledge and resources as nation-backed actors \textbf{(New Zealand’s cyber security strategy 2019)}.
\newline
\newline
Areas of improvement identified by the government are \textbf{(New Zealand’s cyber security strategy 2019)}:

\begin{enumerate}
    \item \textbf{Cyber security aware and active citizens:} Increased regular awareness campaigns and education opportunities for the public in regards to best personal security practices. 
    \item \textbf{Strong and capable cyber security workforce and ecosystem:} Increased promotion and support of the development of the cyber industry in New Zealand.
    \item \textbf{Internationally active:} Detect and prevent any breaches as well as proactively maintaining international relationships regarding information security and participating in any rule reforms. 
    \item \textbf{Resilient and responsive New Zealand:} Supporting infrastructure, businesses, charity organisations, community organisations, individuals in improving security capabilities and resilience.
    \item \textbf{Proactively tackle cybercrime:} Increasing support to impacted parties,  preventing and encouraging reporting of any cybercrimes. 
\end{enumerate}

\par These five principles are planned to continue to improve upon till 2023 \textbf{(New Zealand’s cyber security strategy 2019)}. It is expected that aspects of these will influence the security practices that programmers use in industry. 

\section{Secure Programming Practices}

Secure refers to protecting and deterring any security breaches due to vulnerabilities in program code \textbf{(Requirements of a Better Secure Program Coding)}. Without a focus of security while programming, there is a lack of confidence in the outputted security, resulting in the lack of use and waste of time, effort and money. If unsecured programs are used, it poses significant negative impacts to businesses and individuals. 
\newline
\par The Open Web Application Security Project (OWASP) outlined a checklist to ensure that code is secure. They have provided this document which spans various languages and technologies as the nonprofit foundation aims to improve the security of software. It is stated that it is "much less expensive to build software than to correct security issues". %\textbf{(https://owasp.org/www-pdf-archive/OWASP_SCP_Quick_Reference_Guide_v2.pdf)}
\newline
\newline
Significant items from the checklist include%\textbf{(https://owasp.org/www-pdf-archive/OWASP_SCP_Quick_Reference_Guide_v2.pdf)}: 

\begin{itemize}
\item \textbf{Input Validation:} This is the act of validating that all inputs are trusted. Examples of inputs are data from databases, file streams, etc, as well as client provided data. If anything is classed as 'untrusted', it should fail and result in a rejection. The application should have one input validation routine, a specified common character set and all validation should occur on a preset trusted system. 
\item \textbf{Output Encoding:} When data is sent back to a client, the information needs to be output encoded. To make this secure, a started test routine has to be utilised, and similar to input validation; all encoding must occur on a preset trusted system. 
\item \textbf{Authentication and Password Management:} This is verifying whether a user is allowed to perform an action. Best practice is restricting all resources except the ones intended to be public. All authentication controls should be able to fail in order to maintain security and passwords should be hashed.  In order to pass, authentication has to occur first and every detail has to match the protected records. 
\item \textbf{Session Management:} A session should only associate the same client ID. It can only occur after authentication. 
\item \textbf{Access Control:} Set users access based on system permissions. It should use only preset trusted systems. 
\item \textbf{Cryptographic Practices:} All assets should be protected by cryptography. A policy should be established in how to manage the public and private keys. 
\item \textbf{Error Handing and Logging:} All application errors must be ‘caught’ and handled. They should not disclose any information in the error responses, and instead only show in the logs. The logs themselves can only be accessed by a few and mechanisms should be in place to analyse the logs. 
\item \textbf{Data Protection:} Protect most assets, communications and caches. Encrypt any stored information that holds significant value. 
\item \textbf{Communication Security:} Encrypt any communication channels. Certificates should be valid and the protocol Transport Layer Security should be used. 
\item \textbf{System Configuration:} Ensure that everything is up to date. This includes servers, frameworks, components, languages, IDE's and libraries. Remove unnecessary information before deployment; test code, TODO's, HTTP methods and response headers. 
\item \textbf{Database Security:} Maintain short connections. Authentication needs to be checked prior use, and access control needs to be restricted. 
\item \textbf{File Management:} Require authentication and access control before any interaction with files. Validate file headers. Implement safe uploading by scanning for any malicious intent; viruses and malware. 
\item \textbf{Memory Management:} Check buffers. If larger than expected, potentially has malware or viruses. Avoid the use of vulnerable functions - \textit{printf, strcpy}
\item \textbf{General Coding Practices:}
\begin{itemize}
\item Use tested and trusted components for tasks
\item Restrict users from altering code in any way
\item Review all third part components; code and libraries
\item Initialise all variables and fields
\end{itemize}
\end{itemize}

\section{Related Work}

\subsection{“Think secure from the beginning” – Assal and Chiasson}

In “A Survey with Software Developers” \textbf{(INSERT LINK)}, the authors pursue to understand the human behaviours and motivations which surround factors of software security. The authors specified a series of questions targeted towards software developers through an online survey. The research examined responses to further support the professional development of programmers, both in theory and practice. This aligns well with the research goals of “Why Do Programmers Do What They Do?”, as it aims to form a theory on the influences and effects of decisions surrounding technical work. 
\newline
\newline
The results outlined the following common groups:

\begin{enumerate}
\item \textbf{Work Motivation:} Developers did not lack motivation in their job. They performed based on self-determination.
\item \textbf{Understanding of software security:} Developers had a sound understanding of software security. They grasped the importance of securing technical work and the discussed various methods of doing so and specifying at what stages in the project life-cycle they should implement these based on “best practices”.
\item \textbf{Security Issues:} Majority of the participants believed their software could be compromised, despite being comfortable with the approaches to protecting the software. The majority has also experienced a security issue, whether that be a breach, or vulnerable code.
\end{enumerate}

\par From these common groups the overarching theme displayed was that the developers were not purposefully ignorant to maintaining security practices, the majority were proactive and willing to learn. However, it was the importance of functionality and lack of ongoing support from organisations which made working towards a more secure software challenging. 
\newline
\par This paper was valuable to read as there are strong comparisons between the research topic developed in this and the subject of the 489 topic. The methodologies are different, however, the findings of this display a programmers personal perspective rather than a theorised view. It is a direct comparison to our project, and we can further outline questions directed to the future work stated in the article's conclusion; “to explore potential relationships between motivations, deterrents and strategies for software security” and “investigate security procedures and attitudes in companies that have experienced security breaches and compare it to others who have not”.

